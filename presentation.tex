\documentclass{beamer}

\usepackage[utf8]{inputenc}
\usepackage[T1]{fontenc}
\usepackage{default}
\usepackage{nicefrac}
\usepackage{microtype}

\begin{document}

\title{Thousands of plots} %Modeling optical tweezers}
\author{J. \'Sl\k{e}zak,  M. Eskelinen, P. Andersson, M. Cornely, D. Salgado, C. Amorino, D. Fidanov}

\begin{frame}
\maketitle
\end{frame}

\begin{frame}
We found many folders containing cryptic .txt files. We tried plotting the numbers in one of them as a time series:
\centering
\includegraphics[width=0.7\textwidth, angle=-90, origin=c]{img/time_series_xy.eps}
\end{frame}

\begin{frame}
Let's look at the distributions of values for the two variables:
\centering
\includegraphics[width=0.7\textwidth, angle=-90, origin=c]{img/histograms_xy.eps}
\end{frame}

\begin{frame}
It looks a bit Gaussian. What say Kolmogorov and Smirnov?

For variable 1, normality is not very probable, but not excluded:
\[
\text{p-value} = 0.08741
\]

For variable 2, it's very abnormal:
\[
\text{p-value} = 1.972e-05
\]

Let's look at the density anyway\dots
\end{frame}

\begin{frame}
Pretty close, we'll roll with it.
\centering
\includegraphics[height=\textwidth, angle=-90, origin=c]{img/densities_xy.eps}
\end{frame}

\begin{frame}
Wait, what about correlation between the two variables? Let's do a line plot:

\centering
\includegraphics[height=0.8\textwidth, angle=-90, origin=c, trim={2cm 1cm 5cm 2cm}, clip=false]{img/trajectory.eps}\\
Not much relation, but it's not completely random either. Let's compare\dots
\end{frame}

\begin{frame}
\begin{figure}
\centering
\includegraphics[height=0.8\textwidth, angle=-90, origin=c, trim={6cm 4cm 18cm 2cm}, clip=false]{img/comparison_2000pts.eps}
\caption{Left: First 2000 pairs. Right: 2000 random pairs}
\end{figure}
The jumps between the consecutive points are small compared to actual random variables. Hints of dynamics\dots

\end{frame}

\begin{frame}
If there are dynamics, we should see the later values depending on the earlier ones. Can we measure it somehow?

\begin{block}{Yes!}<2->
For a variable $X$ with mean $\mu$ and variance $\sigma^2$, we can define the autocorrelation as a function of time as
\[
 \mathrm{acor}(\tau) = \frac{E[(X_t - \mu)(X_{t+\tau} - \mu)]}{\sigma^2}.
\]
This tells us the dependence between values of $X$ that are time $\tau$ apart. The filename mentions a frequency of 5kHz, so the time difference between the points is probably 0.2 ms.
\end{block}

\end{frame}

\begin{frame}
\centering
\includegraphics[height=\textwidth, angle=-90, origin=c, trim={2cm 1cm 8cm 1cm}, clip=false]{img/acf_xy.eps}\\

Noise would only have a peak at 0, so we have something! Let's check these in log scale\dots
\end{frame}

\begin{frame}
\centering
\includegraphics[height=\textwidth, angle=-90, origin=c, trim={2cm 1cm 8cm 1cm}, clip=false]{img/log_acf_xy.eps}\\

The dependence between values seems to decay exponentially\\ (for a while\dots)
\end{frame}

\begin{frame}
Since the data is Gaussian, the only stochastic process (dynamic) which produces this decay is the \emph{Ornstein-Uhlenbeck process},
\[
 \frac{\mathrm dx(t)}{\mathrm dt} = - k x(t) + \eta(t).
\]
Here
\begin{itemize}
 \item $k$ is the stiffness of a harmonic force
 \item $\eta$ is white Gaussian noise
\end{itemize}
\end{frame}

\begin{frame}
The solution of the equation is
\[
 x(t) = \int_{-\infty}^t e^{-k (t-s)}\mathrm dB(s),
\]
with $B$ denoting Brownian motion ($\mathrm dB(s) = \eta(s)\mathrm ds$).

We find the autocorrelation for this process to be
%\[
\begin{align*}
 \mathrm{acor}(x(\tau)) &= \frac{E[x(0)x(\tau)]}{\sigma^2} \\
  &=\dots \\
  &= e^{-kt}.
\end{align*}

\uncover<2->{\emph{We can fit this to the data to determine $k$!}}
\end{frame}

\begin{frame}
\centering
\includegraphics[height=0.5\textwidth, angle=-90, origin=c, trim={1cm 1cm 1cm 1cm}, clip=false]{img/fit_actual.eps}

The fit gives us the values of $k$ for variables V1 and V2
\begin{align*}
 k_{V1} &= 106.00,\\
 k_{V2} &= 119.25.
\end{align*}

\end{frame}

\begin{frame}
How do we know if the $k$ we find is right? \emph{Modelling!}

We can integrate the Ornstein-Uhlenbeck process between two consecutive points $\Delta t$ apart to derive a formula for simulation:

\begin{align*}
x_{n+1} &= \alpha x_n + z_n,
\end{align*}
where
\begin{align*}
 \alpha = e^{-k \Delta t}
\end{align*}
and $z_n$ is a white Gaussian noise with variance $(1-\alpha^2) \sigma^2$.
\end{frame}

\begin{frame}
Let's calculate everything from the model using the $k$ values we found:
\centering
\includegraphics[height=\textwidth, angle=-90, origin=c, trim={1cm 1cm 1cm 1cm}, clip=false]{img/everything_modeled.eps}\\

\end{frame}

\begin{frame}
For the simulated data, the linear fit results in the $k$ values
\begin{align*}
 k_{V1} &= 109.37,\\
 k_{V2} &= 138.61,
\end{align*}
which mean that for the first and second variables we are within 4\% and 17\% of the actual values in the simulation, respectively.

\begin{block}{Conclusion:}<2->
The data can be modeled quite well as an Ornstein-Uhlenbeck process!
\end{block}

\end{frame}

\end{document}
